{
\def\sym#1{\ifmmode^{#1}\else\(^{#1}\)\fi}
\begin{tabular}{l*{2}{c}}
\hline\hline
                    &\multicolumn{2}{c}{Mobility}               \\\cmidrule(lr){2-3}
                    &\multicolumn{1}{c}{(1)}         &\multicolumn{1}{c}{(2)}         \\
\hline
Ln distance to military facility $\times$ County Imp. Years (1973-1976) $\times$&     -0.0559         &                     \\
                    &    (0.0293)         &                     \\
[1em]
Ln distance to military facility $\times$ County Imp. Years (1973-1976) $\times$&                     &     -0.0388         \\
                    &                     &    (0.0239)         \\
[1em]
Ln distance to military facility $\times$ County Imp. Years (1973-1976) $\times$&                     &     -0.0761         \\
                    &                     &    (0.0410)         \\
\hline
\# observations     &        7222         &        7222         \\
\# counties         &         319         &         319         \\
Mean outcome        &      0.6898         &      0.6898         \\
\hline\hline \multicolumn{3}{p{25cm}}{\footnotesize * p$<$0.05, ** p$<$0.01, *** p$<$0.001. Note: Unit of observation: county $\times$ week. This table reports OLS coefficient estimates of exposure to state repression over mobility during critical periods. Column (1) reports the main coefficient of interest from equation (\ref{eq:4}), which includes county and week fixed-effects. Column (2) reports the main coefficients of interest from a variation of equation (\ref{eq:4}) which also distinguishes the start of the second critical period. All estimations use the number of years between 18 and 25 lived between 1973 and 1976 as impressionable years indicators and the logarithmic distance to the closest military facility as a proxy for state repression. Observations are weighted by county population size. Conley standard errors in parentheses.}\\ \end{tabular} } % Generated on 19 Sep 2023 at 08:58:22.
